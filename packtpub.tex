
\documentclass[a4paper,
                             twoside,
                             BCOR1.0cm,
                             DIV11,
                             parskip=full,
                             11pt]{scrbook}
\usepackage[standardsections]{scrhack}
\usepackage[T1]{fontenc}
\usepackage[utf8]{inputenc}
\usepackage{xspace}
\usepackage{bera}
\usepackage{pifont}
\usepackage{amssymb}
\usepackage[dvipsnames]{xcolor}
\usepackage{graphicx}
\usepackage{pgf}
\usepackage{tikz}
\usetikzlibrary{shapes}
\usepackage{color}

%%%%%%%%%%%%

\DeclareFixedFont{\numcap}{T1}{phv}{bx}{n}{3cm}
\DeclareFixedFont{\textcap}{T1}{phv}{bx}{n}{1.5cm}
\DeclareFixedFont{\textaut}{T1}{phv}{bx}{n}{0.8cm} 

\addtokomafont{chapter}{\color{gray}\textcap}
\addtokomafont{section}{\color{white}}
\addtokomafont{subsection}{\color{white}}
\setkomafont{pagehead}{\sffamily\small}
\setkomafont{captionlabel}{\sffamily\small\bfseries}
\setkomafont{caption}{\sffamily\small}
%%%%%%%%%%%%%%%%%%%%%%%%%%%%%%%%%%%%%%%%%%%%%%
\usetikzlibrary{calc,trees,positioning,arrows,chains,shapes.geometric,
    decorations.pathreplacing,decorations.pathmorphing,shapes,
    matrix,shapes.symbols}

\tikzset{
  punktchain/.style={
    rectangle, 
    rounded corners, 
    draw=black!20, thin,
    minimum height=3em, 
    text centered},
  peu/.style={
    rectangle,
    fill opacity=1,
    %rounded corners, 
    fill=white,
    top color=white,
    draw=black!20, thin,
    %text width=10em, 
    %minimum height=3em, 
    text centered},
  line/.style={draw, thin, <-},
  element/.style={
    tape,
    top color=white,
    bottom color=blue!50!black!60!,
    minimum width=8em,
    draw=blue!40!black!90, very thick,
    text width=10em, 
    minimum height=3.5em, 
    text centered, 
    on chain},
}
%%%%%%%%%%%%%%%%%%%%%%%%%%%%%%%%%%%%%%%%%%%%%%
\usepackage{scrlayer-scrpage}
\setlength{\headheight}{25pt}
\pagestyle{scrheadings}
\setheadwidth{textwithmarginpar}
\setheadsepline{.4pt}
\addtokomafont{headsepline}{\color{lightgray}}

\lefoot{\color{black!40}{\hrulefill}}
\cefoot{\parbox[c][.5in][c]{1cm}{\fcolorbox{black!40}{white}{\thepage}}}
\refoot{}

\lofoot{\color{black!40}{\hrulefill}}
\cofoot[{\color{black!40}{---}} {\thepage} {\color{black!40}{---}}]{\parbox[c][.5in][c]{1cm}{\fcolorbox{black!40}{white}{\thepage}}}
\rofoot[]{}

\usepackage[pdftex,             
    colorlinks=true,
    linkcolor=blue,
    filecolor=blue,
    citecolor=blue,
    pdftitle={Llibre amb estil},
    pdfauthor={Joan Queralt Gil},
    pdfsubject={tema},
    pdfkeywords={Keyword1, Keyword 2},
    bookmarks, bookmarksnumbered=true]{hyperref}

\tolerance=4000
\emergencystretch=20pt

\setcounter{secnumdepth}{3}
\usepackage{titlesec}

\titleformat{\chapter}[display]
    {\usekomafont{sectioning} \usekomafont{chapter}\filleft}
    {\numcap\textcolor[named]{gray}\thechapter}
    {1em}
    {}

\titleformat{\section}[block]
    {\usekomafont{sectioning}\usekomafont{section}
     \tikz[overlay]  \fill[color=black,rounded corners=.2ex] (0,-1ex) rectangle (\textwidth-2cm,1em);}
    { \thesection}
    {1em}
    {}

\titleformat{\subsection}[block]
    {\usekomafont{sectioning}\usekomafont{subsection}
       \tikz[overlay] \fill[color=black!60] (0,-1ex) rectangle (\textwidth-2cm,1em);}
    { \thesubsection}
    {1em}
    {}

\usepackage{lipsum}
%%%%%%%%%%%%%%%%%%%%
\usepackage{enumitem}

\newlist{steps}{enumerate}{4}
\setlist[steps]{topsep=0pt,partopsep=0pt,itemsep=0pt,parsep=0pt,labelindent=0.5cm,leftmargin=*}
\setlist[steps,1]{label*=\arabic*.}
\setlist[steps,2]{label*=\arabic*.}
\setlist[steps,3]{label*=\arabic*.}
\setlist[steps,4]{label*=\arabic*.}

\newlist{points}{itemize}{4}
\setlist[points]{topsep=0pt,partopsep=0pt,itemsep=0pt,parsep=0pt,labelindent=0.5cm,leftmargin=*}
\setlist[points,1]{label=\tiny\ding{110}}
\setlist[points,2]{label=\tiny\ding{108}}
\setlist[points,3]{label=\tiny\ding{72}}
\setlist[points,4]{label=\tiny\ding{117}}

\newlist{objectives}{itemize}{1}
\setlist[objectives]{topsep=0pt,partopsep=0pt,itemsep=0pt,parsep=0pt,labelindent=0.5cm,leftmargin=*}
\setlist[objectives,1]{label=\tiny$\blacktriangleright$}

\newlist{attention}{itemize}{1}
\setlist[attention]{topsep=0pt,partopsep=0pt,itemsep=0pt,parsep=0pt,labelindent=0.5cm,leftmargin=*}
\setlist[attention,1]{label=\ding{224}}


\newlist{arrows}{itemize}{4}
\setlist[arrows]{topsep=0pt,partopsep=0pt,itemsep=0pt,parsep=0pt,labelindent=0.5cm,leftmargin=*}
\setlist[arrows,1]{label=\tiny\ding{252}}
\setlist[arrows,2]{label=\tiny\ding{212}}
\setlist[arrows,3]{label=\tiny\ding{232}}
\setlist[arrows,4]{label=\tiny\ding{217}}
%%%%%%%%%%%%%%%%%%%%
\usepackage[tikz]{bclogo}
\newcommand\novaimatge{\includegraphics[width=14pt]{write}}
\renewcommand\logowidth{14pt}

\usepackage{colortbl}
\arrayrulecolor{gray}
\let\shline\hline
\def\hline{\noalign{\vskip3pt}\shline\noalign{\vskip4pt}}
%%%%%%%%%%%%%%%%%%%%%%%%%%%%%%%%%%%%%%%%%%%%%%%%%%%%%%%%%%
\begin{document}

%%%%%%%%%%%%%%%%%%%%%% First Page
\title{\textcap{Latex Template}}
\author{
    \textaut{Joan Queralt Gil}\\http://catalatex.blogspot.com/
}
\date{\today}

\maketitle
%%%%%%%%%%%%%%%%%%%%%%
\tableofcontents

%*************************************************************************
\chapter{Format of chapters}\label{cap:primer}
%*************************************************************************
\dictum%
[Timothy K. Paine]%author
{Find an interesting quote, \dots
} %text

This entry, which is located on the right side of the page, is called \textit {Sentence} and is obtained with the command
 \verb+\dictum[author]{text}+.
%*************************************************************************
\section{Titles}\label{sec:titles}
%*************************************************************************
The package \verb+titlesec+ has been used to finish formatting section headings. The command that allows us to do so is:
\begin{scriptsize}
\begin{verbatim}
\titleformat{command}[shape]%
  {format}%
  {label}%
  {sep}% horizontal separation between label and body of the title
  {before}[after]% precedent / next code in the title body
 

\end{verbatim}
\end{scriptsize}

%*******************
\subsection{Chapter Titles}\label{sbsec:captit}
%*******************
The chapters begin with the number and the name in gray letter, which is obtained with the command:
\begin{scriptsize}
\begin{verbatim}
\addtokomafont{chapter}{\color{gray}\textcap}   % gives size to the title of the document
\end{verbatim}
\end{scriptsize}
The font is Helvetica and of different size, the number (\verb+\numcap+) of 3 cm and the text (\verb+\textcap+) of 1.5 cm, format that is obtained with the initial definition:

\begin{scriptsize}
\begin{verbatim}
\DeclareFixedFont{\numcap}{T1}{phv}{bx}{n}{3cm}   %creates big chapter tag numbers
\DeclareFixedFont{\textcap}{T1}{phv}{bx}{n}{1.5cm} %create big letters name of the chapter
\end{verbatim}
\end{scriptsize}
The package \verb+titlesec+ has been used to finish formatting section headings. Specifically for the chapters is defined:
\begin{scriptsize}
\begin{verbatim}
 \titleformat{\chapter}[display]%              
    {\usekomafont{sectioning} \usekomafont{chapter}\filleft}% 
                     %Formatted according to definitions of KOMA and aligns right
    {\numcap\textcolor[named]{gray}\thechapter}%                   
                     % format the color capitol number
    {1em}%
    {}
\end{verbatim}
\end{scriptsize}

%*******************
\subsection{Section Titles}\label{sbsec:sectit}
%*******************
Section headings have the white text on a color background, the color is obtained with the command:
\begin{scriptsize}
\begin{verbatim}
\addtokomafont{section}{\color{white}}
\end{verbatim}
\end{scriptsize}
and the color of black background is achieved with the mentioned package \verb+titlesec+ and the use of a background image generated by the package
\begin{tiny}
\begin{verbatim}
\titleformat{\section}[block]%              
    {\usekomafont{sectioning}\usekomafont{section}% 
     \tikz[overlay]  \fill[color=black,rounded corners=.2ex] (0,-1ex) rectangle (\textwidth-2cm,1em);}%  
    { \thesection}%                   
    {1em}%
    {}
\end{verbatim}
\end{tiny}


%*******************
\subsection{Subsection Title}\label{sbsec:subsectit}
%*******************
The subsection titles have the white text on a color background, the color of the text, as in the case of the sections, is obtained with the command:
\begin{scriptsize}
\begin{verbatim}
\addtokomafont{subsection}{\color{white}}
\end{verbatim}
\end{scriptsize}
and the black background color is achieved with the package \verb+titlesec+ and the use of a background image generated by the package \verb+\tikz+:
\begin{tiny}
\begin{verbatim}
\titleformat{\subsection}[block]%              
    {\usekomafont{sectioning}\usekomafont{subsection}% 
       \tikz[overlay] \fill[color=black!60] (0,-1ex) rectangle (\textwidth-2cm,1em);}%  
    {\thesubsection}%                   
    {1em}%
    {}
\end{verbatim}
\end{tiny}

%*************************************************************************
\section{Page Style} \label{sec:pagstyle}
%*************************************************************************
The pages use the style \verb+scrheadings+ defined by KOMA and customized so that:

\begin{objectives}%[labelindent=20pt,leftmargin=*]
\item  The heading consists of the chapter name on the left page and section on the right page, with a small black sans serif underlined with a gray line that goes beyond the width of the text.
\item  The foot consists of the centered page number and in a gray box that stands above a gray line that goes from side to side of the text.
\end{objectives}
This is achieved by using the \verb+scrpage2+ package of the KOMA set, where it is defined:
\begin{tiny}
\begin{verbatim}
\setlength{\headheight}{25pt}	      		   
\pagestyle{scrheadings}			         % page style
\setheadwidth{textwithmarginpar}	    	% lengthens the header
\setheadsepline{.4pt}					%line under the header
\addtokomafont{headsepline}{\color{lightgray}}% gives gray line under header
%left page footer:
\lefoot{\color{black!40}{\hrulefill}}
\cefoot{\parbox[c][.5in][c]{1cm}{\fcolorbox{black!40}{white}{\thepage}}}
\refoot{}
\lofoot{\color{black!40}{\hrulefill}}
\cofoot[{\color{black!40}{---}} {\thepage} {\color{black!40}{---}}]%
     {\parbox[c][.5in][c]{1cm}{\fcolorbox{black!40}{white}{\thepage}}}
\rofoot[]{}
\end{verbatim}
\end{tiny}
Chapter start pages, in style \verb+plain+, are defined in the preceding commands as the option of the corresponding command, for example:
\begin{scriptsize}\verb+\cfoot[style scrplain ]{style scrheadings}+\end{scriptsize}


%*************************************************************************
\chapter{Improvements}\label{cap:improvements}
%*************************************************************************

Different packages have been used to achieve improvements in different aspects of the body of the text. For example, for the lists the package \verb+enumerate+, for the tables the package \verb+colortbl+ or the package \verb+bclogo+ for the calls, whose details will be seen in the next sections.

%*************************************************************************
\section{Lists}\label{sec:lists}
%*************************************************************************
With the package \verb+enumitem+ we can define new styles of lists with the command:
\begin{tiny}
\begin{verbatim}
\newlist{listname}{tipus=enumerate,itemize,description}{number of levels of nesting} 
\setlist[listname]{format}
\setlist[listname,1]{label=format etiqueta1}
\end{verbatim}
\end{tiny}

In this document we have predetermined a series of lists that we think can be useful.

%*******************
\subsection{Numbered Lists}\label{sbsec:listnum}
%*******************
\subsubsection{Steps}\label{ssbsec:steps}
%******
It is a compact list numbered in the scheme (1. - 1.1 - 1.1.1) indented to the left 0.5 cm to indicate steps. The code to get it is:
\begin{tiny}
\begin{verbatim}
\newlist{steps}{enumerate}{4}
\setlist[steps]{topsep=0pt,partopsep=0pt,itemsep=0pt,parsep=0pt,labelindent=0.5cm,leftmargin=*}
\setlist[steps,1]{label*=\arabic*.}
\setlist[steps,2]{label*=\arabic*.}
\setlist[steps,3]{label*=\arabic*.}
\setlist[steps,4]{label*=\arabic*.}
\end{verbatim}
\end{tiny}
and the result this:
 \begin{steps}
\item first
\item second
\item third
\item fourth
\item fifth
\item sixth
\item seventh
\end{steps}
%*********************
\subsubsection{With numbers from the pifont package}\label{ssbsec:pifont}
%******
This list uses predefined numbers in the package \verb+pifont+ and which are quite interesting. However it can not be defined with the \verb+\newlist+ command and the code must be indicated each time it is used in the body of the text.

The code to generate it is this:
\begin{scriptsize}
\begin{verbatim}
\begin{enumerate}[nolistsep,label=\ding{\value{enumi}},start=202]
\end{verbatim}
\end{scriptsize}
where 202 is the character code of the package \verb+pifont+. See the result:

%% this environment is not allowed to be defined as a \ newlist. You have to put the options in the middle of the doc

\minisec{Starting with character 172 \ding{172}}
\begin{enumerate}[nolistsep,label=\ding{\value{enumi}},start=172]
\item first
\item second
\item third
\item fourth
\item fifth
\item sixth
\item seventh
\item eigth
\item ninth
\item tenth
\end{enumerate}
\minisec{Starting with character 182 \ding{182}}
\begin{enumerate}[nolistsep,label=\ding{\value{enumi}},start=182]
\item first
\item second
\item third
\item fourth
\item fifth
\item sixth
\item seventh
\item eigth
\item ninth
\item tenth
\end{enumerate}

\minisec{Starting with character 202 \ding{202}}

\begin{enumerate}[nolistsep,label=\ding{\value{enumi}},start=202]
\item first
\item second
\item third
\item fourth
\item fifth
\item sixth
\item seventh
\item eigth
\item ninth
\item tenth
\end{enumerate}

%*******************
\subsection{Bulleted lists}\label{sbsec:bullist}
%*******************
\subsubsection{Objectives}\label{ssbsec:objs}
%******
It is a compact list with vignettes corresponding to a small black triangle (it is necessary to call the package \verb+\usepackage {amssymb}+) of a single level of depth indented to the left 0.5 cm to indicate objectives. The code to get it is:
\begin{tiny}
\begin{verbatim}
\newlist{objectives}{itemize}{1}
\setlist[objectives]{topsep=0pt,partopsep=0pt,itemsep=0pt,parsep=0pt,labelindent=0.5cm,leftmargin=*}
\setlist[objectives,1]{label=\tiny$\blacktriangleright$}
\end{verbatim}
\end{tiny}
and the result this:

\begin{objectives}
\item  first objective
\item second objective
\item third objective
\item fourth objective
\end{objectives}

\subsubsection{Attention}\label{ssbsec:at}
%******
It is a compact list with bullets corresponding to an arrow (the character 224 of the package \ verb + pifont +) of a single depth level indented to the left 0.5 cm to indicate entries to be taken into account. The code to get it is:
\begin{tiny}
\begin{verbatim}
\newlist{atencio}{itemize}{1}
\setlist[atencio]{topsep=0pt,partopsep=0pt,itemsep=0pt,parsep=0pt,labelindent=0.5cm,leftmargin=*}
\setlist[atencio,1]{label=\ding{224}}
\end{verbatim}
\end{tiny}
and the result this:

\begin{attention}
\item  first point
\item second point
\item third point
\item fourth point
\end{attention}

\subsubsection{Points}\label{ssbsec:points}
%******
It is a compact list with vignettes corresponding to various types of points (the characters 110, 108, 72 and 117 of the package \verb+pifont+), of 4 levels of depth, indented to the left 0.5 cm to indicate nested entries to keep in mind. The code to get it is:

\begin{tiny}
\begin{verbatim}
\newlist{points}{itemize}{4}
\setlist[points]{topsep=0pt,partopsep=0pt,itemsep=0pt,parsep=0pt,labelindent=0.5cm,leftmargin=*}
\setlist[points,1]{label=\tiny\ding{110}}
\setlist[points,2]{label=\tiny\ding{108}}
\setlist[points,3]{label=\tiny\ding{72}}
\setlist[points,4]{label=\tiny\ding{117}}
\end{verbatim}
\end{tiny}
and the result this:
\begin{points}
\item flying devices
	\begin{points}
		\item biplanes
		\item jets
		\item of transport
			\begin{points}
				\item of a single engine
					\begin{points}
						\item to reaction
						\item to helix
					\end{points}
				\item different motors
			\end{points}
		\item helicopters
	\end{points}
\item automobiles
	\begin{points}
		\item racing cars
		\item private cars
		\item trucks
\end{points}
\item bicycles
\end{points}

\subsubsection{Arrows}\label{ssbsec:arrows}
%******
It is a compact list with vignettes corresponding to various types of arrows (252, 212, 232 and 217 characters of the package \verb+pifont+), with 4 levels of depth, indented to the left 0.5 cm to indicate nested entries to keep in mind. The code to get it is:
\begin{tiny}
\begin{verbatim}
\newlist{arrows}{itemize}{4}
\setlist[arrows]{topsep=0pt,partopsep=0pt,itemsep=0pt,parsep=0pt,labelindent=0.5cm,leftmargin=*}
\setlist[arrows,1]{label=\tiny\ding{252}}
\setlist[arrows,2]{label=\tiny\ding{212}}
\setlist[arrows,3]{label=\tiny\ding{232}}
\setlist[arrows,4]{label=\tiny\ding{217}}
\end{verbatim}
\end{tiny}
and the result this:

\begin{arrows}
\item flying devices
	\begin{arrows}
		\item biplanes
		\item jets
		\item of transport
			\begin{arrows}
				\item of a single engine
					\begin{arrows}
						\item to reaction
						\item to helix
					\end{arrows}
				\item different motors
			\end{arrows}
		\item helicopters
	\end{arrows}
\item automobiles
	\begin{arrows}
		\item racing cars
		\item private cars
		\item trucks
\end{arrows}
\item bicycles
\end{arrows}

\section{Verses and Citations}\label{sec:quote}
%*************************************************************************
\subsection{Poetry}\label{sbsec:poetry}
%*******************

We can write poetry using the \verb+verse+ environment that bleeds to the left and also to the right. To terminate the end of a verse two counter-turns are used: \verb+\\+ i to separate a stanza from the following we can leave more space (with \verb+\bigskip+) or less, with \verb+\medskip+.

\begin{verse}
Touching head in both sides, \\*
moving forward along the water path, \\*
the cow is alone. It's blind. \\*
\medskip

From a rocket thrown with too much trace, \\*
the vailet cleared her an eye, and in the other \\*
He has been given a phone: the cow is blind. \\*
\medskip

Go over to the source as usual, \\*
but not with the stubbornness of other times \\*
nor with your companions, no: comes alone. \\*
\medskip

Their companions, by the cliffs, by the commas, \\*
due to the silence of the meadows and the riverside, \\*
they make the skeleton cling while grazing \\*
fresh grass at random \ dots She would fall. \\*
\medskip

Muzzle top in the crushed sink \\*
and faces back \ dots But returns, \\*
and lower your head to the water, and drink calm. \\*
\medskip

Drink little, not very thirsty. Then raise \\*
to the sky, huge, the embroidered forehead \\*
with a great tragic gesture; blinks \\*
Over the dead dolls, and it returns \\*
orphan of light under the sun that burns, \\*
hesitating for unforgettable paths, \\*
Flaking the long tail gently. \\*
\medskip
\begin{flushright}
Joan Maragall\\
Poesies, 1895\\
\end{flushright}
\end{verse}

\subsection{Citacions}\label{sbsec:citations}
%*******************
Citation can be done with two different environments:
\subsubsection{quote}\label{sbsec:quote}
%*******************

\begin{quote}
There are people who do not like to speak, write or think in Catalan. It is the same people who do not like to talk, write or think.

Ovidi Montllor
\end{quote}

\subsubsection{quotation}\label{sbsec:quotation}
%*******************
\begin{quotation}
There are people who do not like to speak, write or think in Catalan. It is the same people who do not like to talk, write or think.

Ovidi Montllor
\end{quotation}

\section{Minisec}\label{sec:minisec}
%*************************************************************************
Sometimes you want a header that is easily distinguished but that is very close to the text, without too much vertical separation. The command \verb+\minisec+ of the Koma-Script package creates this type of heading without any structural level within the document. This minisection does not produce an entry in the Table of Contents nor has any numbering.

\minisec{Minisec command of Koma-Script}
\lipsum[1-2]

\section{Tables}\label{sec:tables}
%*************************************************************************

Although there is a typographical law that says that vertical lines do not have to be placed in the tables, a very interesting effect can be achieved if these vertical lines are not long enough to touch the upper and lower horizontal lines, a cell

Observe the result:

\begin{table}[h]
\centering
\begin{tabular}{l|c|c|l}
\hline
Day & Max Temperature ºC & Temperature min. C & Observations \\
\hline
Monday & 18 & 8 & Clouds and clearings. \\
Tuesday & 17 & 9 & Clouds and clearings. \\
Wednesdays & 10 & 4 & Clear and windy. \\
Thursday & 8 & 2 & Tramuntana wind, force 5. \\
Friday & 6 & -2 & Clear and shallow wind. \\
Saturday & 4 & -3 & Clouds and clearings. \\
Sunday & 9 & 2 & Clear and shallow wind. \\
\hline
\end{tabular}
\caption{Table of weekly temperatures}\label{tab:temp}
\end{table}

\begin{table}[h]
\centering
\begin{tabular}{c|c|c}
\hline
First column and First column and First column \\
\hline
$a_{1,1}$ & $a_{1,2}$ & $a_{1,3}$ \\
$a_{2,1}$ & $a_{2,2}$ & $a_{2,3}$ \\
$a_{3,1}$ & $a_{3,2}$ & $a_{3,3}$ \\
\hline
\end{tabular}
\caption{Table of ordered items}\label{tab:ordre}
\end{table}

We do this by using the \verb+colortbl+ package that allows you to color the tables. Then we define the color of the lines, in particular we make them gray with \verb+\arrayrulecolor+, and finally shorten the height of the lines:

\begin{verbatim}
\usepackage{colortbl}
\arrayrulecolor{gray}
\let\shline\hline
\def\hline{\noalign{\vskip3pt}\shline\noalign{\vskip4pt}}
\end{verbatim}

\section{Frames}\label{sec:frames}
%*************************************************************************
With the help of the package \verb+bclogo+ you can create boxes and frames with an image or logo, a title and the body of the text.

The code used is this:

\begin{verbatim}
\usepackage[tikz]{bclogo}
\newcommand\novaimatge{\includegraphics[width=14pt]{write}}
\renewcommand\logowidth{14pt}
\end{verbatim}

where \verb+write+ is the name of the image file that appears to the left as the logo of the frame.

In the body of the frame you can put the desired text, even lists such as those defined in the section ~\nameref {sec: lists} on page ~\pageref {sec: lists}.

Let's see an example:

\begin{bclogo}[logo=\novaimatge,couleur=gray!30,barre=none,noborder=true,marge=10,ombre=true,couleurOmbre=black!60,blur]%
{\sffamily{  Objectives}}
\sffamily\scriptsize
%\lipsum[1]
\begin{objectives}
\item  first objective
\item second objective
\item third objective
\item fourth objective
\end{objectives}
\end{bclogo}


\chapter{Show}\label{cap:show}
\dictum%
[Albert Einstein]%author
{There are two infinite things: the Universe and human stupidity. And I'm not sure about the Universe.
} %text
\lipsum[2-3]
\begin{bclogo}[logo=\novaimatge,couleur=gray!30,barre=none,noborder=true,marge=10,ombre=true,couleurOmbre=black!60,blur]%
{\sffamily{  Objectives}}
\sffamily\scriptsize
%\lipsum[1]
\begin{arrows}
\item  first objective
\item second objective
\item third objective
\item fourth objective
\end{arrows}
\end{bclogo}


\lipsum[4]
%*************************************************************************
\section{Fourth section}\label{sec:fourth}
%*************************************************************************
\lipsum[1]
\begin{steps}
\item  first
\item  second
\item  third
\item  fourth
\item  fifth
\item  sixth
\item  seventh
\end{steps}
\subsection{First subsection of the fourth section}\label{sbsec:first}
\lipsum[1-3]
\begin{quote}
There are people who do not like to speak, write or think in Catalan. It is the same people who do not like to talk, write or think.

Ovidi Montllor
\end{quote}
\subsection{Second subsection of the fourth section}\label{sbsec:second}
\minisec{Minisec Command of Koma-Script}
\lipsum[4-5]
\begin{table}[h]
\centering\scriptsize
\begin{tabular}{l|c|c|l}
\hline
Day & Max Temperature ºC & Temperature min. C & Observations \\
\ hline
Monday & 18 & 8 & Clouds and clearings. \\
Tuesday & 17 & 9 & Clouds and clearings. \\
Wednesdays & 10 & 4 & Clear and windy. \\
Thursday & 8 & 2 & Tramuntana wind, force 5. \\
Friday & 6 & -2 & Clear and shallow wind. \\
Saturday & 4 & -3 & Clouds and clearings. \\
Sunday & 9 & 2 & Clear and shallow wind. \\
\hline
\end{tabular}
\caption{Table of weekly temperatures}\label{tab:temp2}
\end{table}
\lipsum[8-9]
\end{document}



\begin{tiny}
\begin{verbatim}

\end{verbatim}
\end{tiny}





\titleformat{\section}[block]%              
    {\usekomafont{sectioning}\usekomafont{section}%
     \tikz[overlay]\shade[left color=blue!20,right color=white](0,-1ex)rectangle(\textwidth,1em);}%    
    {\thesection}%                   
    {1em}%
    {}
    
    
    
         \begin{pgfonlayer}{background}
            \shade[left color=blue!20,right color=white]let \p1=(counter.north),\p2=(content.north)in            (0,{max(\y1,\y2)})rectangle(content.south east);
        \end{pgfonlayer}
