
\documentclass[a4paper,
                             twoside,
                             BCOR1.0cm,
                             DIV11,
                             parskip=full,
                             11pt]{scrbook}

\usepackage[T1]{fontenc}
\usepackage[utf8]{inputenc}
\usepackage{xspace}
\usepackage{bera}
\usepackage{pifont}
\usepackage{amssymb}
\usepackage[dvipsnames]{xcolor}
\usepackage{graphicx}
\usepackage{pgf}
\usepackage{tikz}
\usetikzlibrary{shapes}
\usepackage{color}

%%%%%%%%%%%%

\DeclareFixedFont{\numcap}{T1}{phv}{bx}{n}{3cm}
\DeclareFixedFont{\textcap}{T1}{phv}{bx}{n}{1.5cm}
\DeclareFixedFont{\textaut}{T1}{phv}{bx}{n}{0.8cm} 

\addtokomafont{chapter}{\color{gray}\textcap}
\addtokomafont{section}{\color{white}}
\addtokomafont{subsection}{\color{white}}
\setkomafont{pagehead}{\sffamily\small}
\setkomafont{captionlabel}{\sffamily\small\bfseries}
\setkomafont{caption}{\sffamily\small}
%%%%%%%%%%%%%%%%%%%%%%%%%%%%%%%%%%%%%%%%%%%%%%
\usetikzlibrary{calc,trees,positioning,arrows,chains,shapes.geometric,
    decorations.pathreplacing,decorations.pathmorphing,shapes,
    matrix,shapes.symbols}

\tikzset{
  punktchain/.style={
    rectangle, 
    rounded corners, 
    draw=black!20, thin,
    minimum height=3em, 
    text centered},
  peu/.style={
    rectangle,
    fill opacity=1,
    %rounded corners, 
    fill=white,
    top color=white,
    draw=black!20, thin,
    %text width=10em, 
    %minimum height=3em, 
    text centered},
  line/.style={draw, thin, <-},
  element/.style={
    tape,
    top color=white,
    bottom color=blue!50!black!60!,
    minimum width=8em,
    draw=blue!40!black!90, very thick,
    text width=10em, 
    minimum height=3.5em, 
    text centered, 
    on chain},
}
%%%%%%%%%%%%%%%%%%%%%%%%%%%%%%%%%%%%%%%%%%%%%%
\usepackage{scrpage2}
\setlength{\headheight}{25pt}
\pagestyle{scrheadings}
\setheadwidth{textwithmarginpar}
\setheadsepline{.4pt}
\addtokomafont{headsepline}{\color{lightgray}}

\lefoot{\color{black!40}{\hrulefill}}
\cefoot{\parbox[c][.5in][c]{1cm}{\fcolorbox{black!40}{white}{\thepage}}}
\refoot{}

\lofoot{\color{black!40}{\hrulefill}}
\cofoot[{\color{black!40}{---}} {\thepage} {\color{black!40}{---}}]{\parbox[c][.5in][c]{1cm}{\fcolorbox{black!40}{white}{\thepage}}}
\rofoot[]{}

\usepackage[pdftex,             
    colorlinks=true,
    linkcolor=blue,
    filecolor=blue,
    citecolor=blue,
    pdftitle={Llibre amb estil},
    pdfauthor={Joan Queralt Gil},
    pdfsubject={tema},
    pdfkeywords={Keyword1, Keyword 2},
    bookmarks, bookmarksnumbered=true]{hyperref}

\tolerance=4000
\emergencystretch=20pt

\setcounter{secnumdepth}{3}
\usepackage{titlesec}

\titleformat{\chapter}[display]
    {\usekomafont{sectioning} \usekomafont{chapter}\filleft}
    {\numcap\textcolor[named]{gray}\thechapter}
    {1em}
    {}

\titleformat{\section}[block]
    {\usekomafont{sectioning}\usekomafont{section}
     \tikz[overlay]  \fill[color=black,rounded corners=.2ex] (0,-1ex) rectangle (\textwidth-2cm,1em);}
    { \thesection}
    {1em}
    {}

\titleformat{\subsection}[block]
    {\usekomafont{sectioning}\usekomafont{subsection}
       \tikz[overlay] \fill[color=black!60] (0,-1ex) rectangle (\textwidth-2cm,1em);}
    { \thesubsection}
    {1em}
    {}

\usepackage{lipsum}
%%%%%%%%%%%%%%%%%%%%
\usepackage{enumitem}

\newlist{steps}{enumerate}{4}
\setlist[steps]{topsep=0pt,partopsep=0pt,itemsep=0pt,parsep=0pt,labelindent=0.5cm,leftmargin=*}
\setlist[steps,1]{label*=\arabic*.}
\setlist[steps,2]{label*=\arabic*.}
\setlist[steps,3]{label*=\arabic*.}
\setlist[steps,4]{label*=\arabic*.}

\newlist{points}{itemize}{4}
\setlist[points]{topsep=0pt,partopsep=0pt,itemsep=0pt,parsep=0pt,labelindent=0.5cm,leftmargin=*}
\setlist[points,1]{label=\tiny\ding{110}}
\setlist[points,2]{label=\tiny\ding{108}}
\setlist[points,3]{label=\tiny\ding{72}}
\setlist[points,4]{label=\tiny\ding{117}}

\newlist{objectives}{itemize}{1}
\setlist[objectives]{topsep=0pt,partopsep=0pt,itemsep=0pt,parsep=0pt,labelindent=0.5cm,leftmargin=*}
\setlist[objectives,1]{label=\tiny$\blacktriangleright$}

\newlist{attention}{itemize}{1}
\setlist[attention]{topsep=0pt,partopsep=0pt,itemsep=0pt,parsep=0pt,labelindent=0.5cm,leftmargin=*}
\setlist[attention,1]{label=\ding{224}}


\newlist{arrows}{itemize}{4}
\setlist[arrows]{topsep=0pt,partopsep=0pt,itemsep=0pt,parsep=0pt,labelindent=0.5cm,leftmargin=*}
\setlist[arrows,1]{label=\tiny\ding{252}}
\setlist[arrows,2]{label=\tiny\ding{212}}
\setlist[arrows,3]{label=\tiny\ding{232}}
\setlist[arrows,4]{label=\tiny\ding{217}}
%%%%%%%%%%%%%%%%%%%%
\usepackage[tikz]{bclogo}
\newcommand\novaimatge{\includegraphics[width=14pt]{escriu}}
\renewcommand\logowidth{14pt}

\usepackage{colortbl}
\arrayrulecolor{gray}
\let\shline\hline
\def\hline{\noalign{\vskip3pt}\shline\noalign{\vskip4pt}}
%%%%%%%%%%%%%%%%%%%%%%%%%%%%%%%%%%%%%%%%%%%%%%%%%%%%%%%%%%
\begin{document}

%%%%%%%%%%%%%%%%%%%%%% First Page
\title{\textcap{Latex Template}}
\author{
    \textaut{Joan Queralt Gil}\\http://phobos.xtec.cat/jqueralt
}
\date{\today}

\maketitle
%%%%%%%%%%%%%%%%%%%%%%
\tableofcontents

%*************************************************************************
\chapter{Format of chapters}\label{cap:primer}
%*************************************************************************
\dictum%
[Timothy K. Paine]%author
{Find an interesting quote, \dots
} %text

This entry, which is located on the right side of the page, is called \textit {Sentence} and is obtained with the command
 \verb+\dictum[author]{text}+.
%*************************************************************************
\section{Titles}\label{sec:titles}
%*************************************************************************
The package \verb+titlesec+ has been used to finish formatting section headings. The command that allows us to do so is:
\begin{scriptsize}
\begin{verbatim}
\titleformat{command}[shape]%
  {format}%
  {label}%
  {sep}% horizontal separation between label and body of the title
  {before}[after]% precedent / next code in the title body
 

\end{verbatim}
\end{scriptsize}

%*******************
\subsection{Chapter Titles}\label{sbsec:captit}
%*******************
The chapters begin with the number and the name in gray letter, which is obtained with the command:
\begin{scriptsize}
\begin{verbatim}
\addtokomafont{chapter}{\color{gray}\textcap}   % gives size to the title of the document
\end{verbatim}
\end{scriptsize}
The font is Helvetica and of different size, the number (\verb+\numcap+) of 3 cm and the text (\verb+\textcap+) of 1.5 cm, format that is obtained with the initial definition:

\begin{scriptsize}
\begin{verbatim}
\DeclareFixedFont{\numcap}{T1}{phv}{bx}{n}{3cm}   %creates big chapter tag numbers
\DeclareFixedFont{\textcap}{T1}{phv}{bx}{n}{1.5cm} %create big letters name of the chapter
\end{verbatim}
\end{scriptsize}
The package \verb+titlesec+ has been used to finish formatting section headings. Specifically for the chapters is defined:
\begin{scriptsize}
\begin{verbatim}
 \titleformat{\chapter}[display]%              
    {\usekomafont{sectioning} \usekomafont{chapter}\filleft}% 
                     %Formatted according to definitions of KOMA and aligns right
    {\numcap\textcolor[named]{gray}\thechapter}%                   
                     % format the color capitol number
    {1em}%
    {}
\end{verbatim}
\end{scriptsize}

%*******************
\subsection{Section Titles}\label{sbsec:sectit}
%*******************
Section headings have the white text on a color background, the color is obtained with the command:
\begin{scriptsize}
\begin{verbatim}
\addtokomafont{section}{\color{white}}
\end{verbatim}
\end{scriptsize}
and the color of black background is achieved with the mentioned package \verb+titlesec+ and the use of a background image generated by the package
\begin{tiny}
\begin{verbatim}
\titleformat{\section}[block]%              
    {\usekomafont{sectioning}\usekomafont{section}% 
     \tikz[overlay]  \fill[color=black,rounded corners=.2ex] (0,-1ex) rectangle (\textwidth-2cm,1em);}%  
    { \thesection}%                   
    {1em}%
    {}
\end{verbatim}
\end{tiny}


%*******************
\subsection{Subsection Title}\label{sbsec:subsectit}
%*******************
The subsection titles have the white text on a color background, the color of the text, as in the case of the sections, is obtained with the command:
\begin{scriptsize}
\begin{verbatim}
\addtokomafont{subsection}{\color{white}}
\end{verbatim}
\end{scriptsize}
and the black background color is achieved with the package \verb+titlesec+ and the use of a background image generated by the package \verb+\tikz+:
\begin{tiny}
\begin{verbatim}
\titleformat{\subsection}[block]%              
    {\usekomafont{sectioning}\usekomafont{subsection}% 
       \tikz[overlay] \fill[color=black!60] (0,-1ex) rectangle (\textwidth-2cm,1em);}%  
    {\thesubsection}%                   
    {1em}%
    {}
\end{verbatim}
\end{tiny}

%*************************************************************************
\section{Page Style} \label{sec:pagstyle}
%*************************************************************************
The pages use the style \verb+scrheadings+ defined by KOMA and customized so that:

\begin{objectives}%[labelindent=20pt,leftmargin=*]
\item  The heading consists of the chapter name on the left page and section on the right page, with a small black sans serif underlined with a gray line that goes beyond the width of the text.
\item  The foot consists of the centered page number and in a gray box that stands above a gray line that goes from side to side of the text.
\end{objectives}
This is achieved by using the \verb+scrpage2+ package of the KOMA set, where it is defined:
\begin{tiny}
\begin{verbatim}
\setlength{\headheight}{25pt}	      		   
\pagestyle{scrheadings}			         % page style
\setheadwidth{textwithmarginpar}	    	% lengthens the header
\setheadsepline{.4pt}					%line under the header
\addtokomafont{headsepline}{\color{lightgray}}% gives gray line under header
%left page footer:
\lefoot{\color{black!40}{\hrulefill}}
\cefoot{\parbox[c][.5in][c]{1cm}{\fcolorbox{black!40}{white}{\thepage}}}
\refoot{}
\lofoot{\color{black!40}{\hrulefill}}
\cofoot[{\color{black!40}{---}} {\thepage} {\color{black!40}{---}}]%
     {\parbox[c][.5in][c]{1cm}{\fcolorbox{black!40}{white}{\thepage}}}
\rofoot[]{}
\end{verbatim}
\end{tiny}
Chapter start pages, in style \verb+plain+, are defined in the preceding commands as the option of the corresponding command, for example:
\begin{scriptsize}\verb+\cfoot[style scrplain ]{style scrheadings}+\end{scriptsize}


%*************************************************************************
\chapter{Improvements}\label{cap:improvements}
%*************************************************************************

Different packages have been used to achieve improvements in different aspects of the body of the text. For example, for the lists the package \verb+enumerate+, for the tables the package \verb+colortbl+ or the package \verb+bclogo+ for the calls, whose details will be seen in the next sections.

%*************************************************************************
\section{Lists}\label{sec:lists}
%*************************************************************************
With the package \verb+enumitem+ we can define new styles of lists with the command:
\begin{tiny}
\begin{verbatim}
\newlist{listname}{tipus=enumerate,itemize,description}{number of levels of nesting} 
\setlist[listname]{format}
\setlist[listname,1]{label=format etiqueta1}
\end{verbatim}
\end{tiny}

In this document we have predetermined a series of lists that we think can be useful.

%*******************
\subsection{Numbered Lists}\label{sbsec:listnum}
%*******************
\subsubsection{Steps}\label{ssbsec:steps}
%******
Es tracta d'una llista compacta numerada en esquema (1. - 1.1 - 1.1.1 ) sagnada a l'esquerra 0.5 cm per indicar passos. El codi per obtenir-la és:
\begin{tiny}
\begin{verbatim}
\newlist{steps}{enumerate}{4}
\setlist[steps]{topsep=0pt,partopsep=0pt,itemsep=0pt,parsep=0pt,labelindent=0.5cm,leftmargin=*}
\setlist[steps,1]{label*=\arabic*.}
\setlist[steps,2]{label*=\arabic*.}
\setlist[steps,3]{label*=\arabic*.}
\setlist[steps,4]{label*=\arabic*.}
\end{verbatim}
\end{tiny}
i el resultat aquest:
\begin{steps}
\item first
\item second
\item third
\item fourth
\item fifth
\item sixth
\item seventh
\end{steps}
%*********************
\subsubsection{Amb números del paquet pifont}\label{ssbsec:pifont}
%******
Aquesta llista utilitza com a números els predefinits al paquet \verb+pifont+ i que són força interessants. Tanmateix no es pot definir amb el comandament \verb+\newlist+ i cal indicar-ne el codi cada cop que s'utilitza al cos del text.

El codi per generar-la és aquest:
\begin{scriptsize}
\begin{verbatim}
\begin{enumerate}[nolistsep,label=\ding{\value{enumi}},start=202]
\end{verbatim}
\end{scriptsize}
on 202 és el codi del caràcter del paquet \verb+pifont+. Vegeu-ne el resultat:

%%aquest entorn no es deixa definir com a \newlist. Cal posar les opcions al mig del doc

\minisec{Començant pel caràcter 172 \ding{172}}
\begin{enumerate}[nolistsep,label=\ding{\value{enumi}},start=172]
\item first
\item second
\item third
\item fourth
\item fifth
\item sixth
\item seventh
\item eigth
\item ninth
\item tenth
\end{enumerate}
\minisec{Començant pel caràcter 182 \ding{182}}
\begin{enumerate}[nolistsep,label=\ding{\value{enumi}},start=182]
\item  primer
\item segon
\item tercer
\item quart
\item cinquè
\item sisè
\item setè
\item vuitè
\item novè
\item desè
\end{enumerate}

\minisec{Començant pel caràcter 202 \ding{202}}

\begin{enumerate}[nolistsep,label=\ding{\value{enumi}},start=202]
\item  primer
\item segon
\item tercer
\item quart
\item cinquè
\item sisè
\item setè
\item vuitè
\item novè
\item desè
\end{enumerate}

%*******************
\subsection{Llistes amb vinyetes}\label{sbsec:llistpics}
%*******************
\subsubsection{Objectives}\label{ssbsec:objs}
%******
Es tracta d'una llista compacta amb vinyetes corresponents a un petit triangle negre (cal cridar el paquet \verb+\usepackage{amssymb}+) d'un sol nivell de profunditat  sagnada a l'esquerra 0.5 cm per indicar objectius. El codi per obtenir-la és:
\begin{tiny}
\begin{verbatim}
\newlist{objectives}{itemize}{1}
\setlist[objectives]{topsep=0pt,partopsep=0pt,itemsep=0pt,parsep=0pt,labelindent=0.5cm,leftmargin=*}
\setlist[objectives,1]{label=\tiny$\blacktriangleright$}
\end{verbatim}
\end{tiny}
i el resultat aquest:

\begin{objectives}
\item  first objective
\item second objective
\item third objective
\item fourth objective
\end{objectives}

\subsubsection{Atenció}\label{ssbsec:at}
%******
Es tracta d'una llista compacta amb vinyetes corresponents a una fletxa (el caràcter 224 del paquet \verb+pifont+) d'un sol nivell de profunditat  sagnada a l'esquerra 0.5 cm per indicar entrades a tenir en compte. El codi per obtenir-la és:
\begin{tiny}
\begin{verbatim}
\newlist{atencio}{itemize}{1}
\setlist[atencio]{topsep=0pt,partopsep=0pt,itemsep=0pt,parsep=0pt,labelindent=0.5cm,leftmargin=*}
\setlist[atencio,1]{label=\ding{224}}
\end{verbatim}
\end{tiny}
i el resultat aquest:

\begin{attention}
\item  first point
\item second point
\item third point
\item fourth point
\end{attention}

\subsubsection{Points}\label{ssbsec:points}
%******
Es tracta d'una llista compacta amb vinyetes corresponents a diversos tipus de punts (els caràcters 110, 108, 72 i 117 del paquet \verb+pifont+), de 4 nivells de profunditat,  sagnada a l'esquerra 0.5 cm per indicar  entrades niuades a tenir en compte. El codi per obtenir-la és:
\begin{tiny}
\begin{verbatim}
\newlist{points}{itemize}{4}
\setlist[points]{topsep=0pt,partopsep=0pt,itemsep=0pt,parsep=0pt,labelindent=0.5cm,leftmargin=*}
\setlist[points,1]{label=\tiny\ding{110}}
\setlist[points,2]{label=\tiny\ding{108}}
\setlist[points,3]{label=\tiny\ding{72}}
\setlist[points,4]{label=\tiny\ding{117}}
\end{verbatim}
\end{tiny}
i el resultat aquest:
\begin{points}
\item aparells voladors
	\begin{points}
		\item biplans
		\item jets
		\item de transport
			\begin{points}
				\item d'un sol motor
					\begin{points}
						\item a reacció
						\item a hèlix
					\end{points}
				\item diferents motors
			\end{points}
		\item helicòpters
	\end{points}
\item automòbils
	\begin{points}
		\item cotxes de carreres
		\item cotxes privats
		\item camions
\end{points}
\item bicicletes
\end{points}

\subsubsection{Arrows}\label{ssbsec:arrows}
%******
Es tracta d'una llista compacta amb vinyetes corresponents a diversos tipus de fletxa (els caràcters 252, 212, 232 i 217 del paquet \verb+pifont+), de 4 nivells de profunditat,  sagnada a l'esquerra 0.5 cm per indicar  entrades niuades a tenir en compte. El codi per obtenir-la és:
\begin{tiny}
\begin{verbatim}
\newlist{arrows}{itemize}{4}
\setlist[arrows]{topsep=0pt,partopsep=0pt,itemsep=0pt,parsep=0pt,labelindent=0.5cm,leftmargin=*}
\setlist[arrows,1]{label=\tiny\ding{252}}
\setlist[arrows,2]{label=\tiny\ding{212}}
\setlist[arrows,3]{label=\tiny\ding{232}}
\setlist[arrows,4]{label=\tiny\ding{217}}
\end{verbatim}
\end{tiny}
i el resultat aquest:
\begin{arrows}
\item aparells voladors
	\begin{arrows}
		\item biplans
		\item jets
		\item de transport
			\begin{arrows}
				\item d'un sol motor
					\begin{arrows}
						\item a reacció
						\item a hèlix
					\end{arrows}
				\item diferents motors
			\end{arrows}
		\item helicòpters
	\end{arrows}
\item automòbils
	\begin{arrows}
		\item cotxes de carreres
		\item cotxes privats
		\item camions
\end{arrows}
\item bicicletes
\end{arrows}

\section{Versos i Citacions}\label{sec:quote}
%*************************************************************************
\subsection{Poesia}\label{sbsec:poesia}
%*******************

Podem escriure poesia utilitzant l'entorn \verb+verse+ que sagna per l'esquerra i també per la dreta. Per dterminar el final d'un vers s'utilitzen dues contrabarres: \verb+\\+ i per separar una estrofa de la següent podem deixar més espai (amb \verb+\bigskip+) o menys, amb  \verb+\medskip+.


\begin{verse}

Topant de cap en una i altra soca,\\*
avançant d'esma pel camí de l'aigua,\\*
se'n ve la vaca tota sola. És cega.\\*
\medskip

D'un cop de roc llançat amb massa traça,\\*
el vailet va buidar-li un ull, i en l'altre\\*
se li ha posat un tel: la vaca és cega.\\*
\medskip

Ve a abeurar-se a la font com ans solia,\\*
mes no amb el posat ferm d'altres vegades\\*
ni amb ses companyes, no: ve tota sola.\\*
\medskip

Ses companyes, pels cingles, per les comes,\\*
pel silenci dels prats i en la ribera,\\*
fan dringar l'esquellot mentre pasturen\\*
l'herba fresca a l'atzar\dots Ella cauria.\\*
\medskip

Topa de morro en l'esmolada pica\\*
i recula afrontada\dots Però torna,\\*
i abaixa el cap a l'aigua, i beu calmosa.\\*
\medskip

Beu poc, sens gaire set. Després aixeca\\*
al cel, enorme, l'embanyada testa\\*
amb un gran gesto tràgic; parpelleja\\*
damunt les mortes nines, i se'n torna\\*
orfe de llum sota el sol que crema,\\*
vacil·lant pels camins inoblidables,\\*
brandant llànguidament la llarga cua.\\*
\medskip
\begin{flushright}
Joan Maragall\\
Poesies, 1895\\
\end{flushright}
\end{verse}

\subsection{Citacions}\label{sbsec:citacio}
%*******************
Les citacions es poden fer amb dos entorns diferents:
\subsubsection{quote}\label{sbsec:quote}
%*******************

\begin{quote}
Hi ha gent a qui no agrada que es parle, s’escriga o es pense en català. És la mateixa gent a qui no els agrada que es parle, s’escriga o es pense.

Ovidi Montllor
\end{quote}

\subsubsection{quotation}\label{sbsec:quotation}
%*******************
\begin{quotation}
Hi ha gent a qui no agrada que es parle, s’escriga o es pense en català. És la mateixa gent a qui no els agrada que es parle, s’escriga o es pense.

Ovidi Montllor
\end{quotation}

\section{Minisec}\label{sec:minisec}
%*************************************************************************
A vegades es desitja un encapçalament que es distingeixi fàcilment però que estigui molt a prop del text, sense massa separació vertical. El comandament \verb+\minisec+ del paquet Koma-Script crea  aquest tipus d'encapçalaments sense cap nivell estructural dins del document. Aquesta minisecció no produeix una entrada en la Taula de continguts ni té cap numeració. 

\minisec{Comandament minisec de Koma-Script}
\lipsum[1-2]

\section{Taules}\label{sec:taules}
%*************************************************************************

Tot i que hi ha una llei tipogràfica que diu que a les taules no s'hi han de posar línies verticals, es pot aconseguir un efecte força interessant si aquestes línies verticals no són prou llargues com per tocar les línies horitzontals superior i inferior d'un cel·la. 

Observi's el resultat:

\begin{table}[h]
\centering
\begin{tabular}{l|c|c|l}
\hline
Dia & Temperatura màx. ºC & Temperatura mín.ºC & Observacions \\
\hline
dilluns   & 18 & 8 & Núvols i clarianes. \\
dimarts   & 17 & 9 & Núvols i clarianes. \\
dimecres  & 10 & 4 & Clar i ventós. \\
dijous    &  8 & 2 & Vent de tramuntana, força 5. \\
divendres &  6 & -2 & Clar i vent fluix. \\
dissabte  & 4 & -3 & Núvols i clarianes. \\
diumenge  & 9 & 2 &  Clar i vent fluix. \\
\hline
\end{tabular}
\caption{Taula de temperatures setmanals}\label{tab:temp}
\end{table}

\begin{table}[h]
\centering
\begin{tabular}{c|c|c}
\hline
Primera columna & Primera columna & Primera columna \\
\hline
$a_{1,1}$ & $a_{1,2}$ & $a_{1,3}$ \\
$a_{2,1}$ & $a_{2,2}$ & $a_{2,3}$ \\
$a_{3,1}$ & $a_{3,2}$ & $a_{3,3}$ \\
\hline
\end{tabular}
\caption{Taula d'elements ordenats}\label{tab:ordre}
\end{table}

Ho aconseguim amb l'ús del paquet \verb+colortbl+ que permet acolorir les taules. Aleshores definim el color de les línies, en concret les fem de color gris amb \verb+\arrayrulecolor+, i finalment escurcem l'alçada de les línies:

\begin{verbatim}
\usepackage{colortbl}
\arrayrulecolor{gray}
\let\shline\hline
\def\hline{\noalign{\vskip3pt}\shline\noalign{\vskip4pt}}
\end{verbatim}

\section{Marcs}\label{sec:marcs}
%*************************************************************************
Amb l'ajut del paquet \verb+bclogo+ es poden crear caixes i marcs amb una imatge o logo, un títol i el cos del text.

El codi emprat és aquest:

\begin{verbatim}
\usepackage[tikz]{bclogo}
\newcommand\novaimatge{\includegraphics[width=14pt]{escriu}}
\renewcommand\logowidth{14pt}
\end{verbatim}

on \verb+escriu+ és el nom del fitxer d'imatge que apareix a l'esquerra com a logo del marc.

Al cos del marc s'hi pot posar el text que es vulgui, fins i tot llistes com les definides a la secció ~\nameref{sec:llistes} de la pàgina ~\pageref{sec:llistes}.

Vegem-ne un exemple:

\begin{bclogo}[logo=\novaimatge,couleur=gray!30,barre=none,noborder=true,marge=10,ombre=true,couleurOmbre=black!60,blur]%
{\sffamily{  Objectives}}
\sffamily\scriptsize
%\lipsum[1]
\begin{objectives}
\item  first objective
\item second objective
\item third objective
\item fourth objective
\end{objectives}
\end{bclogo}


\chapter{Mostra}\label{cap:mostra}
\dictum%
[Albert Einstein]%author
{Hi ha dues coses infinites: l'Univers i l'estupidesa humana. I de l'Univers no n'estic segur.
} %text
\lipsum[2-3]
\begin{bclogo}[logo=\novaimatge,couleur=gray!30,barre=none,noborder=true,marge=10,ombre=true,couleurOmbre=black!60,blur]%
{\sffamily{  Objectives}}
\sffamily\scriptsize
%\lipsum[1]
\begin{arrows}
\item  first objective
\item second objective
\item third objective
\item fourth objective
\end{arrows}
\end{bclogo}


\lipsum[4]
%*************************************************************************
\section{Quarta secció}\label{sec:quarta}
%*************************************************************************
\lipsum[1]
\begin{steps}
\item  first
\item  second
\item  third
\item  fourth
\item  fifth
\item  sixth
\item  seventh
\end{steps}
\subsection{Primera subsecció de la quarta secció}\label{sbsec:primera}
\lipsum[1-3]
\begin{quote}
Hi ha gent a qui no agrada que es parle, s’escriga o es pense en català. És la mateixa gent a qui no els agrada que es parle, s’escriga o es pense.

Ovidi Montllor
\end{quote}
\subsection{Segona subsecció de la quarta secció}\label{sbsec:segona}
\minisec{Comandament minisec de Koma-Script}
\lipsum[4-5]
\begin{table}[h]
\centering\scriptsize
\begin{tabular}{l|c|c|l}
\hline
Dia & Temperatura màx. ºC & Temperatura mín.ºC & Observacions \\
\hline
dilluns   & 18 & 8 & Núvols i clarianes. \\
dimarts   & 17 & 9 & Núvols i clarianes. \\
dimecres  & 10 & 4 & Clar i ventós. \\
dijous    &  8 & 2 & Vent de tramuntana, força 5. \\
divendres &  6 & -2 & Clar i vent fluix. \\
dissabte  & 4 & -3 & Núvols i clarianes. \\
diumenge  & 9 & 2 &  Clar i vent fluix. \\
\hline
\end{tabular}
\caption{Taula de temperatures setmanals}\label{tab:temp2}
\end{table}
\lipsum[8-9]
\end{document}



\begin{tiny}
\begin{verbatim}

\end{verbatim}
\end{tiny}





\titleformat{\section}[block]%              
    {\usekomafont{sectioning}\usekomafont{section}%
     \tikz[overlay]\shade[left color=blue!20,right color=white](0,-1ex)rectangle(\textwidth,1em);}%    
    {\thesection}%                   
    {1em}%
    {}
    
    
    
         \begin{pgfonlayer}{background}
            \shade[left color=blue!20,right color=white]let \p1=(counter.north),\p2=(content.north)in            (0,{max(\y1,\y2)})rectangle(content.south east);
        \end{pgfonlayer}
